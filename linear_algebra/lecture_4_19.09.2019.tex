\section{Метод Гаусса решения СЛУ (метод исключения неизвестных)}

Дана СЛУ с расширенной матрицей $(A | b)$

Было: элементарные преобразования строк в $(A | b)$ сохраняют множество решений.

\textbf{Алгоритм} 

Прямой ход метода Гаусса

Выполняя элементарные преобразования строк в $(A | b)$, приведем $A$ к ступенчатому виду:

\begin{equation*}
    \begin{pmatrix} 
        0 & \dots & 0 & a_{ij_1} & \dots & \dots & \dots & b_1 \\
        0 & \dots & 0 & 0 & a_{2j_2} & \dots & \dots & b_2 \\
        \vdots & \vdots & \vdots & \vdots & \vdots & \vdots & \vdots & \vdots \\
        0 & 0 & 0 & \dots & 0 & 0 & a_{rj_r} & b_r \\
        0 & 0 & 0 & \dots & 0 & 0 & 0 & b_{r + 1} \\
        0 & 0 & 0 & \dots & 0 & 0 & 0 & 0
    \end{pmatrix} 
.\end{equation*}

\begin{description}
\item[Случай 1] $\exists i \geq r + 1 : b_i \neq 0$

    Тогда в новой СЛУ $i$-е уравнение $0 \cdot x_1 + \dots + 0 \cdot x_n = b_i$, т.е. $0 = b_i \implies $ СЛУ несовместна 

\item[Случай 2] либо $r = m$, либо $b_i = 0 \quad \forall i \geq r + 1$

    Выполняя элементарные преобразования строк приводим матрицу к улучшенному ступенчатому виду -- обратный ход метода Гаусса

    \begin{equation*}
        \begin{pmatrix}
            0 & \dots & 0 & 1 & * & 0 & * & 0 & 0 & * \\
            0 & \dots & 0 & 0 & \dots & 1 & * & 0 & 0 & * \\
            0 & \dots & 0 & 0 & \dots & 0 & 0 & 1 & 0 & * \\
            \vdots & \vdots & \vdots & \vdots & \vdots & \vdots & \vdots & \vdots & \vdots & \vdots \\
            0 & 0 & 0 & \dots & \dots & \dots & \dots & 0 & 1 & * \\
            0 & 0 & 0 & \dots & 0 & 0 & 0 & 0 & 0 & 0
        \end{pmatrix}
    \end{equation*}

    Неизвестные $x_1, x_2, \dots, x_r$ называются \textit{главными}, а остальные \textit{свободными}.

    \begin{description}
    \item[Подслучай 2.1] $r = n$, т.е. все неизвестные -- главные 

        \begin{equation*}
            \begin{pmatrix} 
                1 & \dots & 0 b_1 \\
                \dots & \ddots & \dots & \dots \\
                0 & \dots & 1 & b_r \\
                0 & \dots & 0 & 0
            \end{pmatrix} \leftrightarrow \begin{cases}
                x_1 = b_1 \\
                x_2 = b_2 \\
                \vdots \\
                x_r = b_r
            \end{cases} \text{ -- единственное решение}   
        .\end{equation*}
    \item[Подслучай 2.2] $r < n$, т.е. есть хотя бы одна свободная неизвестная
        
        Перенесем в каждом уравнении все члены со свободными неизвестными в правую часть, получаем выражения всех главных неизвестных через свободные, эти выражения называется \textit{общим решением исходной СЛУ}.
    \end{description}
\end{description}

\begin{example}
    Улучшенный ступенчатый вид:
    \begin{equation*}
        \begin{pmatrix} 
            1 & 3 & 0 & 1 && -1 \\
            0 & 0 & 1 & -2 && 4
        \end{pmatrix} 
    .\end{equation*}

    Главные неизвестные: $x_1, x_3$

    Свободные неизвестные: $x_2, x_4$.

    $x_2 = t_1, x_4 = t_2$ -- параметры.

    \begin{equation*}
       kek % Пример из конспектов
    .\end{equation*}
\end{example}

\begin{consequence}
    Всякая СЛУ с коэффициентами из $\RR$ имеет либо 0 решений, либо одно решение, либо бесконечно много решений.
\end{consequence}

\begin{definition}
    СЛУ называется однородной (ОСЛУ), если все её правые части равны 0. Расширенная: $(A | 0)$ 
\end{definition}

\textbf{Очевидный факт: } Всякая ОСЛУ имеет нулевое решение $(x_1 = x_2 = \dots = x_n = 0)$.

\begin{consequence}
    Всякая ОСЛУ либо имеет ровно 1 решение (нулевое), либо бесконечно много решений.
\end{consequence}

\begin{consequence}
    Всякая ОСЛУ имеет, у которой число неизвестных больше числа уравнений, имеет ненулевое решение
\end{consequence}

\begin{proof}
    В ступенчатом виде будет хотя бы одна свободная неизвестная. Придавая ей ненулевое значение, получим ненулевое решение
\end{proof}

\subsection{} % Придумать название)0

\begin{equation}
    Ax = b \text{, совместная}
    \tag{$\star$}
.\end{equation}

Частное решение СЛУ($\star$) -- это какое то одно её решение.

\begin{statement}
    $Ax = b$ -- совместная СЛУ.

    $x_0$ -- частное решение

    $S \subset \RR^n$ -- множество решений ОСЛУ $Ax = 0$

    $L \subset \RR^n $ -- множество решений $Ax = b$.

    Тогда, $L = x_0 + S$, где $x_0 + S = \{x_0 + v | v \in S\}$
\end{statement}

\begin{proof}~
    \begin{enumerate}
    \item 
        Пусть $u \in L$. 
        Положим $v = u - x_0$

        Тогда $u = x_0 + v$. $Av = A(u - x_0) = Au - Ax_0 = b - b = 0 \implies v \in S \implies L \subset x_0 + S$

    \item
        Пусть $v \in S$, положим $u = x_0 + v$.

        Тогда, $Au = A(x_0 + v) = Ax_0 + Av = b + 0 = b \implies u \in L \implies x_0 + S \subset L$

    \end{enumerate}
\end{proof}

\subsection{Матричные уравнения}

\begin{enumerate}
\item $AX = B$
    A, B известны
    X -- неизвестная матрица

\item $XA = C$
    A, C известны
    X -- неизвестная матрица
\end{enumerate}

2. $\leftrightarrow A^T X = B^T$

\subsubsection{Тип (I)}

$AX = B$

это уравнение равносильно системе
\begin{equation*}
    \begin{cases}
        AX^{(1)} = B^{(1)} \\
        AX^{(2)} = B^{(2)} \\
        \vdots \\
        AX^{(p)} = B^{(p)} \\
    \end{cases}
.\end{equation*}

Этот набор СЛУ надо решать одновременно методом Гаусса

Записываем матрицу $(A|B)$ элементарными преобразованиями строк с ней приводим $A$ к улучшенному ступенчатому виду.

Получаем $(A' | B')$ -- $A'$ имеет улучшенный ступенчатый вид.

Остается выписать общее решение для каждой СЛУ 
\begin{equation*}
    \begin{cases}
        A' x^{(1)} = B^{(1)} \\
        A' x^{(2)} = B^{(2)} \\
        \vdots \\
        A' x^{(p)} = B^{(p)}
    \end{cases}
.\end{equation*}

\subsubsection{Обратные матрицы}

\begin{definition}
    Матрица $B \in M_n$ называется \textit{обратной}, к $A$, если $AB = BA = E$.

    Обозначение: $A^{-1}$
\end{definition}

Факты:
\begin{enumerate}
\item Если $A^{-1} E$, то она определена однозначно
\item Если $AB = E$ для некоторой $B \in M_n$, то $BA = E$ автоматически и тогда $B = A^{-1}$
\end{enumerate}

\begin{consequence}
    $A^{-1}$ является решение матричного уравнения $AX = E$ (если решение существует)
\end{consequence}

\subsection{Перестановки}

\begin{definition}
    \textit{Перестановкой}(или подстановкой) на множестве $\{1, 2, \dots, n\}$ называется всякое биективное (взаимно однозначное) отображение  

    $\sigma : \{1, 2, \dots, n\} \to \{1, 2, \dots, n\}$
\end{definition}
