\section{Семинар 1}

\subsection{Контакты}
Трушин Дмитрий Витальевич -- Дима

trushindima@yandex.ru

\subsection{Матрицы}
\subsubsection{Аномалии}

\begin{enumerate}
\item 
    $A \cdot B \neq B \cdot A$

    $\begin{pmatrix} 0 & 1 \\ 0 & 0 \end{pmatrix} 
    \begin{pmatrix} 0 & 0 \\ 1 & 0 \end{pmatrix}
    =
    \begin{pmatrix} 1 & 0 \\ 0 & 0 \end{pmatrix}$

    $\begin{pmatrix} 0 & 0 \\ 1 & 0 \end{pmatrix}
    \begin{pmatrix} 0 & 1 \\ 0 & 0 \end{pmatrix} 
    =
    \begin{pmatrix} 0 & 0 \\ 0 & 1 \end{pmatrix}$

    $\begin{pmatrix} 1 & 0 \\ 0 & 0 \end{pmatrix} 
    \begin{pmatrix} 0 & 0 \\ 0 & 1 \end{pmatrix} = 0$

    $A \neq 0 \cdot B \neq 0 = 0$

\item 
    $A \neq 0$
    
    $A^2 = 0$

    $\begin{pmatrix} 0 & 0 \\ 1 & 0 \end{pmatrix}^2 = 0$
    
    $\begin{pmatrix} 0 & 1 \\ 0 & 0 \end{pmatrix}^2 = 0$
\end{enumerate}

\subsubsection{Блочные операции}

\begin{tabular}{|c|c|}
    \hline
    A & B \\
    \hline
    C & D \\
    \hline
\end{tabular}

\begin{tabular}{|c|c|}
    \hline
    X & Y \\
    \hline
    Z & W \\
    \hline
\end{tabular}

\begin{tabular}{|c|c|}
    \hline
    $AX + BZ$ & $AY + BW$ \\
    \hline
    $CY + DZ$ & $CY + DW$ \\
    \hline
\end{tabular}


$A \cdot B = A \cdot (B_1, B_2, \dots, B_N) $

$B = (B_1, B_2, \dots, B_N) = (AB_1, AB_2, \dots, AB_n)$


$A = (A_1, \dots, A_n)$

$B = (B_1, \dots, B_n)$

$AB^T = (A_1, \dots, A_n) \begin{pmatrix} B_1^T \\ \vdots \\ B_N^T \end{pmatrix} = A_1 B_1^T + \dots + A_n B_n^T$

\subsubsection{Кек}

% матрица с единицами выше главной диагонали, остальное нули * A = матрица A сдвинутая вверх
% A * эту же матрицу --- ->
% Если же в этой матрице в последний ряд добавить единичку, тогда она при унможении будет циклически сдвигать строки
% А еще так можно можно делать любую перестановку строк и столбцов




\subsubsection{Лол}

$(A \cdot B)^T = B^T \cdot A^T$

\subsubsection{Хех}

$
\begin{tabular}{|c|c|}
    \hline
    A & B \\
    \hline
    C & D \\
    \hline
\end{tabular}^T
$

\begin{tabular}{|c|c|}
    \hline
    $A^T$ & $C^T$ \\
    \hline
    $B^T$ & $D^T$ \\
    \hline
\end{tabular}

\subsubsection{Мда}

% J_0 = матрица с нулями выше диагонали

$x \in M_n(\RR)$

$x J_0 = J_0 x$



